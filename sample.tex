%----------------------------------------------------------------------------------------
%    PACKAGES AND THEMES
%----------------------------------------------------------------------------------------

\documentclass[aspectratio=169,xcolor=dvipsnames]{beamer}
\usetheme{SimpleDarkBlue}
\usepackage[UTF8]{ctex} % 主要中文支持包
\usepackage{fontspec}   % 字体设置
\usepackage{hyperref}
\usepackage{graphicx} % Allows including images
\usepackage{booktabs} % Allows the use of \toprule, \midrule and \bottomrule in tables
\usepackage{tikz}
\usetikzlibrary{shapes,arrows,positioning}

\tikzstyle{block} = [rectangle, draw, rounded corners, minimum height=1.5cm, 
                    text width=3cm, text centered, drop shadow, fill=white]
%----------------------------------------------------------------------------------------
%    TITLE PAGE
%----------------------------------------------------------------------------------------

\title{Empathy in VR}

\author{The Sixth Group}

\date{\today} % Date, can be changed to a custom date

%----------------------------------------------------------------------------------------
%    PRESENTATION SLIDES
%----------------------------------------------------------------------------------------

\begin{document}

\begin{frame}
    \titlepage
\end{frame}

\begin{frame}{Overview}
    \tableofcontents
\end{frame}

% % First Teammate
% \section{Background}

% Second Teammate
\section{Technology}
\begin{frame}{Technology}
    \centering
    \vspace{-0.5cm}
    
    \begin{columns}[onlytextwidth,T]
        \column{0.48\textwidth}
        \begin{block}{\color{white!80!black}存在性 }
 
        \end{block}
        
        \vspace{0.3cm}
        
        \begin{alertblock}{\color{white!70!black}具身VR和全身所有权错觉}

   
        \end{alertblock}
        
        \column{0.48\textwidth}
        \begin{block}{\color{white!80!black}代理错觉 }

 
        \end{block}
        
        \vspace{0.3cm}
        
        \begin{alertblock}{\color{white!80!black}内感知信号操控}


        \end{alertblock}
    \end{columns}
    
    \vspace{0.5cm}
    
    \begin{block}{\color{white!80!black}普罗透斯效应}
 

    \end{block}
    
    \vspace{0.3cm}
    
    \begin{beamercolorbox}[wd=0.9\textwidth,center,sep=4pt,rounded=true,shadow=true]{block body}
        \footnotesize
        这些技术相互配合,共同构建了一个能够促使用户产生共情和利他行为的虚拟环境。\\
        通过营造沉浸感、建立身体所有权、增强代理感、调节情绪以及引导行为,\\
        VR技术展现了其在促进共情能力方面的巨大潜力。
    \end{beamercolorbox}
\end{frame}


\begin{frame}{存在性}
    \begin{enumerate}
        \item 存在性:通过地方错觉和可信度错觉,VR可以让用户感受到身临其境和真实,从而更容易产生共情。
        \begin{itemize}
            \item \textbf{存在性 (Presence):}
            \item 地方错觉 (Place Illusion): 通过高分辨率、大视场角的显示技术,结合3D音频和触觉反馈(如触觉手套、振动平台),营造出身临其境的虚拟环境,让用户感觉像真的置身于虚拟场景中。
            \item 可信度错觉 (Plausibility Illusion): 通过逼真的图形渲染、物理引擎模拟、以及符合现实世界逻辑的交互设计,让虚拟环境的行为和反应看起来真实可信,增强用户的沉浸感。
        \end{itemize}
    \end{enumerate}
\end{frame}

\begin{frame}{具身VR和全身所有权错觉}
    \begin{enumerate}
        \setcounter{enumi}{1}
        \item 具身VR和全身所有权错觉:通过让用户感受到自己拥有一个不同的身体,VR可以让用户从他人视角体验世界,从而更容易理解他人的感受和需求。
        \begin{itemize}
            \item \textbf{具身VR和全身所有权错觉 (Embodiment and Body Ownership Illusion):}
            \item 具身VR (Embodied VR): 使用能够追踪全身动作的传感器(如全身动捕系统、深度摄像头),将用户的动作实时映射到虚拟化身上,使用户能够以虚拟身体的视角在虚拟环境中进行交互。
            \item 全身所有权错觉 (Body Ownership Illusion): 通过视觉、触觉和本体感觉的同步,让用户将虚拟身体感知为自己的身体。例如,当虚拟手接触虚拟物体时,用户的手也同时感受到触觉反馈,从而增强对虚拟身体的拥有感。
        \end{itemize}
    \end{enumerate}
\end{frame}

\begin{frame}{代理错觉}
    \begin{enumerate}
        \setcounter{enumi}{2}
        \item 代理错觉:通过让用户感受到自己对虚拟身体的行动具有控制权,VR可以让用户产生对利他行为的自我归因,从而增强利他行为。
        \begin{itemize}
            \item \textbf{代理错觉 (Agency):}
            \item 动作同步 (Action Synchronization): 精确的动作捕捉和低延迟的渲染技术,确保用户的动作与虚拟化身的行为同步,让用户感觉自己完全控制着虚拟身体。
            \item 因果反馈 (Causal Feedback): 虚拟环境对用户的动作做出符合预期的反应,例如,虚拟手按下按钮后,虚拟设备启动,使用户明确感知到自己的行为对环境产生了影响。
        \end{itemize}
    \end{enumerate}
\end{frame}

\begin{frame}{内感知信号操控}
    \begin{enumerate}
        \setcounter{enumi}{3}
        \item 内感知信号操控:通过操控用户的生理信号,VR可以帮助用户控制情绪,从而将共情转化为同情和利他行为。
        \begin{itemize}
            \item \textbf{内感知信号操控 (Interoceptive Signal Manipulation):}
            \item 生理信号监测 (Physiological Signal Monitoring): 使用生物传感器(如心率监测仪、呼吸传感器、皮肤电传感器)实时监测用户的生理状态。
            \item 生理信号反馈 (Physiological Signal Feedback): 将监测到的生理信号以视觉、听觉或触觉的形式反馈给用户,例如,将心率以虚拟场景中的颜色变化表示,或者通过调整虚拟环境的音乐节奏来反映用户的情绪状态。
            \item 生理信号调节 (Physiological Signal Regulation): 通过生物反馈技术,引导用户调节自己的生理状态,例如,通过控制呼吸来降低心率,从而控制情绪。
        \end{itemize}
    \end{enumerate}
\end{frame}

\begin{frame}{普罗透斯效应}
    \begin{enumerate}
        \setcounter{enumi}{4}
        \item 普罗透斯效应:通过使用具有利他特征的虚拟形象,VR可以影响用户的行为和态度,从而促进共情和利他行为。
        \begin{itemize}
            \item \textbf{普罗透斯效应 (Proteus Effect):}
            \item 虚拟化身定制 (Avatar Customization): 允许用户自定义虚拟化身的外貌、性格和行为特征,特别是赋予其具有利他特征的属性,如帮助他人的行为、友善的表情等。
            \item 行为引导 (Behavioral Priming): 在虚拟环境中设计特定的情境和任务,引导用户扮演具有利他特征的虚拟角色,例如,帮助虚拟角色解决困难、与虚拟角色进行积极的互动等。
        \end{itemize}
    \end{enumerate}
    
    \vspace{5mm}
    
\end{frame}
 \begin{frame}{设计基于VR的共情训练的框架 }
    该框架基于三个关键问题,并提供了相关的共情能力、调节器、催化剂、学习方法和VR技术。旨在帮助教育者设计有效的VR共情训练应用。
     \begin{block}{What is the relationship between emote and observer?}
        情绪化者与观察者之间的关系是什么?
     \end{block}

     \begin{alertblock}{How developed is the self-awareness of the observer?}
     观察者的自我意识是如何发展的?
      \end{alertblock}

     \begin{block}{How developed are the empathic abilities of the observer toward the emote?}
     观察者对情绪化者的同理心能力如何发展?
      \end{block}
     
          
 \end{frame}
 
\begin{frame}{What is the relationship between emote and observer?}
    
    \begin{enumerate}
        \item \textbf{What is the relationship between emote and observer?(情绪化者与观察者之间的关系是什么?)}
        \begin{itemize}
            \item Abilities(能力):群际开放性、反思思维、社交技能、冲突管理。
            \item Catalysts(催化剂):长期培训、安全环境、协作动力、参与志愿活动。
            \item Moderators(调节器):增加对外群体成员的熟悉度、亲和力和相似性;减少偏见、刻板印象、编码预测和对内群体成员的分类思考;增强平等主义目标和自我分析。
            \item Learning methods(学习方法):建构主义和社会情感学习用于反思思维的仪器化;实施平等主义目标;重复启动非刻板印象联想;个体化和否定刻板印象;正念训练用于非评判性思维实践。
            \item EVR methods(虚拟现实方法):增强自我-其他相似性的群体间体现;普罗透斯效应
        \end{itemize}
    \end{enumerate}
\end{frame}

\begin{frame}{How developed is the self-awareness of the observer?}
    \begin{enumerate}
        \setcounter{enumi}{1}
        \item \textbf{How developed is the self-awareness of the observer?(观察者的自我意识是如何发展的?)}
        \begin{itemize}
            \item Abilities(能力):身体、情感、认知和社会自我意识。
            \item Catalysts(催化剂):教育者作为促进者。
            \item Moderators(调节器):自我-他人区分;情绪识别;平等主义的内部和社会目标。
            \item Learning methods(学习方法):正念训练用于内省觉知;实施平等主义目标;心理扫描。
            \item EVR methods(虚拟现实方法):PI和PSI错觉;内感知信号操控。
        \end{itemize}
    \end{enumerate}
\end{frame}

\begin{frame}{How developed are the empathic abilities of the observer toward the emote?}
    \begin{enumerate}
        \setcounter{enumi}{2}
        \item \textbf{How developed are the empathic abilities of the observer toward the emote?(观察者对情绪化者的同理心能力如何发展?)}
        \begin{itemize}
            \item Abilities(能力):情感同理心、认知同理心、同理心准确性、同理心痛苦调节、同情心、利他主义、问题解决。
            \item Catalysts(催化剂):基于真实世界的案例和情境知识。
            \item Moderators(调节器):情感投入、视角转换、在线模拟、对话技巧、当前注意力、仁爱、动机、帮助的力量和技巧;行为表达的自控。
            \item Learning methods(学习方法):角色扮演;正念训练用于当前注意力、视角转换和同情心;实施平等主义目标;心理扫描。
            \item EVR methods(虚拟现实方法):多传感器第一人称视角转换同步性;普罗透斯效应和代理错觉。
        \end{itemize}
    \end{enumerate}
\end{frame}

% % Third Teammate
% \section{Applications}

% % Fourth Teammate
% \section{Evaluation}

% % Fifth Teammate
% \section{Challenges}
% \begin{frame}{Challenges}
    
% \end{frame}


\begin{frame}
\frametitle{Ethical Challenges & Future Directions}
\begin{itemize}
    \item \textbf{VR as the “Ultimate Empathy Machine”:}
    \begin{itemize}
        \item Virtual Reality (VR) has been promoted as a tool to foster empathy, especially for social and humanitarian causes (e.g., refugee crises, racial issues).
        \item Industry claims that VR enhances empathy more effectively than traditional media by providing immersive, embodied experiences.
    \end{itemize}
    \item \textbf{Critical Review:}
    \begin{itemize}
        \item This paper critiques the "VR-empathy" model, arguing that there is insufficient evidence to support VR's universal ability to enhance empathy in the long term.
    \end{itemize}
\end{itemize}
\end{frame}

\begin{frame}
\frametitle{The VR-Empathy Model and Its Issues}
\begin{itemize}
    \item \textbf{Claim of Empathy through Immersion:}
    \begin{itemize}
          \item VR aims to elicit empathy by allowing users to experience another person’s life firsthand (e.g., VR films like \textit{Clouds Over Sidra}). For more details, visit the following link: \href{https://v.qq.com/x/page/d03194nt7hs.html}{Click here to watch}.

        \item Proponents claim that VR fosters pro-social behavior by connecting people emotionally with others' experiences.
    \end{itemize}
    \item \textbf{Empirical Challenges:}
    \begin{itemize}
        \item \textbf{Lack of Evidence:} Studies show mixed results; no substantial proof that VR leads to long-term empathy or motivates pro-social behavior.
        \item \textbf{Bias and Short-Term Effects:} Responses are often influenced by personal biases (e.g., race, gender) and may be short-lived.
    \end{itemize}
\end{itemize}
\end{frame}

\begin{frame}
\frametitle{Key Findings from Empirical Studies}
\begin{itemize}
    \item \textbf{Lack of Long-Term Effects:}
    \begin{itemize}
        \item \textbf{Studies:} Few long-term studies on the impact of VR on empathy, with most studies showing only short-term changes in attitudes.
        \item \textbf{Comparison to Other Media:} Other media (cinema, literature) may be just as effective in fostering empathy.
    \end{itemize}
    \item \textbf{Cultural and Personal Biases:}
    \begin{itemize}
        \item \textbf{Impact of Identity:} Participants’ empathy is influenced by their own social identity and biases, which can distort the effectiveness of VR empathy experiences.
        \item \textbf{Empathy for Similar Groups:} People often feel more empathy for individuals they perceive as similar to themselves (e.g., same race, gender).
    \end{itemize}
\end{itemize}
\end{frame}

\begin{frame}
\frametitle{Ethical Considerations in VR Empathy}
\begin{itemize}
    \item \textbf{Mediated Empathy and Ethical Concerns:}
    \begin{itemize}
        \item VR experiences may induce “empathetic stress,” causing emotional fatigue or discomfort, particularly in vulnerable groups.
        \item \textbf{Risk of Exploitation:} The voyeuristic nature of some VR experiences may objectify those portrayed, creating a false sense of empathy or detachment.
    \end{itemize}
    \item \textbf{Need for Ethical Guidelines:}
    \begin{itemize}
        \item Ethical concerns include audience safety, emotional well-being, and ensuring that VR content is responsibly designed, especially when dealing with sensitive topics.
    \end{itemize}
\end{itemize}
\end{frame}

\begin{frame}
\frametitle{Future Research Directions}
\begin{itemize}
    \item \textbf{Need for Rigorous Research:}
    \begin{itemize}
        \item \textbf{Longitudinal Studies:} More long-term studies are required to assess the lasting impact of VR on empathy.
        \item \textbf{Cultural Sensitivity:} Research should consider how VR experiences are received by different cultural groups and how personal biases affect empathy responses.
    \end{itemize}
    \item \textbf{Design Considerations:}
    \begin{itemize}
        \item VR experiences should not just focus on immersion and empathy but also integrate critical thinking and reflection about the social issues depicted.
        \item Storytelling elements and interactive designs should be carefully crafted to encourage meaningful engagement and empathy.
    \end{itemize}
\end{itemize}
\end{frame}

\begin{frame}
\frametitle{Conclusion}
\begin{itemize}
    \item \textbf{VR’s Potential vs. Reality:}
    \begin{itemize}
        \item While VR has potential to foster empathy, it should not be viewed as a "magic bullet" for social change.
        \item There is insufficient evidence to claim VR as an inherently superior medium for empathy compared to traditional media.
    \end{itemize}
    \item \textbf{Call for More Research:}
    \begin{itemize}
        \item Future studies must rigorously evaluate the long-term effects of VR and develop ethical frameworks for its use in social and humanitarian contexts.
    \end{itemize}
\end{itemize}
\end{frame}

\begin{frame}
\frametitle{Questions}
\begin{itemize}
    \item Thank you for your attention!  
    \item Feel free to ask any questions.
\end{itemize}
\end{frame}

\end{document}
\end{document}

% %------------------------------------------------
% \section{First Section}
% %------------------------------------------------

% \begin{frame}{Bullet Points}
%     \begin{itemize}
%         \item Lorem ipsum dolor sit amet, consectetur adipiscing elit
%         \item Aliquam blandit faucibus nisi, sit amet dapibus enim tempus eu
%         \item Nulla commodo, erat quis gravida posuere, elit lacus lobortis est, quis porttitor odio mauris at libero
%         \item Nam cursus est eget velit posuere pellentesque
%         \item Vestibulum faucibus velit a augue condimentum quis convallis nulla gravida
%     \end{itemize}
% \end{frame}

% %------------------------------------------------

% \begin{frame}{Blocks of Highlighted Text}
%     In this slide, some important text will be \alert{highlighted} because it's important. Please, don't abuse it.

%     \begin{block}{Block}
%         Sample text
%     \end{block}

%     \begin{alertblock}{Alertblock}
%         Sample text in red box
%     \end{alertblock}

%     \begin{examples}
%         Sample text in green box. The title of the block is ``Examples".
%     \end{examples}
% \end{frame}

% %------------------------------------------------

% \begin{frame}{Multiple Columns}
%     \begin{columns}[c] % The "c" option specifies centered vertical alignment while the "t" option is used for top vertical alignment

%         \column{.45\textwidth} % Left column and width
%         \textbf{Heading}
%         \begin{enumerate}
%             \item Statement
%             \item Explanation
%             \item Example
%         \end{enumerate}

%         \column{.45\textwidth} % Right column and width
%         Lorem ipsum dolor sit amet, consectetur adipiscing elit. Integer lectus nisl, ultricies in feugiat rutrum, porttitor sit amet augue. Aliquam ut tortor mauris. Sed volutpat ante purus, quis accumsan dolor.

%     \end{columns}
% \end{frame}

% %------------------------------------------------
% \section{Second Section}
% %------------------------------------------------

% \begin{frame}{Table}
%     \begin{table}
%         \begin{tabular}{l l l}
%             \toprule
%             \textbf{Treatments} & \textbf{Response 1} & \textbf{Response 2} \\
%             \midrule
%             Treatment 1         & 0.0003262           & 0.562               \\
%             Treatment 2         & 0.0015681           & 0.910               \\
%             Treatment 3         & 0.0009271           & 0.296               \\
%             \bottomrule
%         \end{tabular}
%         \caption{Table caption}
%     \end{table}
% \end{frame}

% %------------------------------------------------

% \begin{frame}{Theorem}
%     \begin{theorem}[Mass--energy equivalence]
%         $E = mc^2$
%     \end{theorem}
% \end{frame}

% %------------------------------------------------

% \begin{frame}{Figure}
%     Uncomment the code on this slide to include your own image from the same directory as the template .TeX file.
%     %\begin{figure}
%     %\includegraphics[width=0.8\linewidth]{test}
%     %\end{figure}
% \end{frame}

% %------------------------------------------------

% \begin{frame}[fragile] % Need to use the fragile option when verbatim is used in the slide
%     \frametitle{Citation}
%     An example of the \verb|\cite| command to cite within the presentation:\\~

%     This statement requires citation \cite{p1}.
% \end{frame}

% %------------------------------------------------

% \begin{frame}{References}
%     \footnotesize
%     \bibliography{reference.bib}
%     \bibliographystyle{apalike}
% \end{frame}

% %------------------------------------------------

% \begin{frame}
%     \Huge{\centerline{\textbf{The End}}}
% \end{frame}

% %----------------------------------------------------------------------------------------
