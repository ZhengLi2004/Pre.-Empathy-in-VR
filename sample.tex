%----------------------------------------------------------------------------------------
%    PACKAGES AND THEMES
%----------------------------------------------------------------------------------------

\documentclass[aspectratio=169,xcolor=dvipsnames]{beamer}
\usetheme{SimpleDarkBlue}
\usepackage{hyperref}
\usepackage{graphicx} % Allows including images
\usepackage{booktabs} % Allows the use of \toprule, \midrule and \bottomrule in tables

%----------------------------------------------------------------------------------------
%    TITLE PAGE
%----------------------------------------------------------------------------------------

\title{Empathy in VR}

\author{The Sixth Group}

\date{\today} % Date, can be changed to a custom date

%----------------------------------------------------------------------------------------
%    PRESENTATION SLIDES
%----------------------------------------------------------------------------------------

\begin{document}

\begin{frame}
    \titlepage
\end{frame}

\begin{frame}{Overview}
    \tableofcontents
\end{frame}

% % First Teammate
% \section{Background}

% % Second Teammate
% \section{Technology}

% % Third Teammate
% \section{Applications}

% % Fourth Teammate
% \section{Evaluation}

% % Fifth Teammate
% \section{Challenges}
% \begin{frame}{Challenges}
    
% \end{frame}


\begin{frame}
\frametitle{Ethical Challenges & Future Directions}
\begin{itemize}
    \item \textbf{VR as the “Ultimate Empathy Machine”:}
    \begin{itemize}
        \item Virtual Reality (VR) has been promoted as a tool to foster empathy, especially for social and humanitarian causes (e.g., refugee crises, racial issues).
        \item Industry claims that VR enhances empathy more effectively than traditional media by providing immersive, embodied experiences.
    \end{itemize}
    \item \textbf{Critical Review:}
    \begin{itemize}
        \item This paper critiques the "VR-empathy" model, arguing that there is insufficient evidence to support VR's universal ability to enhance empathy in the long term.
    \end{itemize}
\end{itemize}
\end{frame}

\begin{frame}
\frametitle{The VR-Empathy Model and Its Issues}
\begin{itemize}
    \item \textbf{Claim of Empathy through Immersion:}
    \begin{itemize}
          \item VR aims to elicit empathy by allowing users to experience another person’s life firsthand (e.g., VR films like \textit{Clouds Over Sidra}). For more details, visit the following link: \href{https://v.qq.com/x/page/d03194nt7hs.html}{Click here to watch}.

        \item Proponents claim that VR fosters pro-social behavior by connecting people emotionally with others' experiences.
    \end{itemize}
    \item \textbf{Empirical Challenges:}
    \begin{itemize}
        \item \textbf{Lack of Evidence:} Studies show mixed results; no substantial proof that VR leads to long-term empathy or motivates pro-social behavior.
        \item \textbf{Bias and Short-Term Effects:} Responses are often influenced by personal biases (e.g., race, gender) and may be short-lived.
    \end{itemize}
\end{itemize}
\end{frame}

\begin{frame}
\frametitle{Key Findings from Empirical Studies}
\begin{itemize}
    \item \textbf{Lack of Long-Term Effects:}
    \begin{itemize}
        \item \textbf{Studies:} Few long-term studies on the impact of VR on empathy, with most studies showing only short-term changes in attitudes.
        \item \textbf{Comparison to Other Media:} Other media (cinema, literature) may be just as effective in fostering empathy.
    \end{itemize}
    \item \textbf{Cultural and Personal Biases:}
    \begin{itemize}
        \item \textbf{Impact of Identity:} Participants’ empathy is influenced by their own social identity and biases, which can distort the effectiveness of VR empathy experiences.
        \item \textbf{Empathy for Similar Groups:} People often feel more empathy for individuals they perceive as similar to themselves (e.g., same race, gender).
    \end{itemize}
\end{itemize}
\end{frame}

\begin{frame}
\frametitle{Ethical Considerations in VR Empathy}
\begin{itemize}
    \item \textbf{Mediated Empathy and Ethical Concerns:}
    \begin{itemize}
        \item VR experiences may induce “empathetic stress,” causing emotional fatigue or discomfort, particularly in vulnerable groups.
        \item \textbf{Risk of Exploitation:} The voyeuristic nature of some VR experiences may objectify those portrayed, creating a false sense of empathy or detachment.
    \end{itemize}
    \item \textbf{Need for Ethical Guidelines:}
    \begin{itemize}
        \item Ethical concerns include audience safety, emotional well-being, and ensuring that VR content is responsibly designed, especially when dealing with sensitive topics.
    \end{itemize}
\end{itemize}
\end{frame}

\begin{frame}
\frametitle{Future Research Directions}
\begin{itemize}
    \item \textbf{Need for Rigorous Research:}
    \begin{itemize}
        \item \textbf{Longitudinal Studies:} More long-term studies are required to assess the lasting impact of VR on empathy.
        \item \textbf{Cultural Sensitivity:} Research should consider how VR experiences are received by different cultural groups and how personal biases affect empathy responses.
    \end{itemize}
    \item \textbf{Design Considerations:}
    \begin{itemize}
        \item VR experiences should not just focus on immersion and empathy but also integrate critical thinking and reflection about the social issues depicted.
        \item Storytelling elements and interactive designs should be carefully crafted to encourage meaningful engagement and empathy.
    \end{itemize}
\end{itemize}
\end{frame}

\begin{frame}
\frametitle{Conclusion}
\begin{itemize}
    \item \textbf{VR’s Potential vs. Reality:}
    \begin{itemize}
        \item While VR has potential to foster empathy, it should not be viewed as a "magic bullet" for social change.
        \item There is insufficient evidence to claim VR as an inherently superior medium for empathy compared to traditional media.
    \end{itemize}
    \item \textbf{Call for More Research:}
    \begin{itemize}
        \item Future studies must rigorously evaluate the long-term effects of VR and develop ethical frameworks for its use in social and humanitarian contexts.
    \end{itemize}
\end{itemize}
\end{frame}

\begin{frame}
\frametitle{Questions}
\begin{itemize}
    \item Thank you for your attention!  
    \item Feel free to ask any questions.
\end{itemize}
\end{frame}

\end{document}
\end{document}

% %------------------------------------------------
% \section{First Section}
% %------------------------------------------------

% \begin{frame}{Bullet Points}
%     \begin{itemize}
%         \item Lorem ipsum dolor sit amet, consectetur adipiscing elit
%         \item Aliquam blandit faucibus nisi, sit amet dapibus enim tempus eu
%         \item Nulla commodo, erat quis gravida posuere, elit lacus lobortis est, quis porttitor odio mauris at libero
%         \item Nam cursus est eget velit posuere pellentesque
%         \item Vestibulum faucibus velit a augue condimentum quis convallis nulla gravida
%     \end{itemize}
% \end{frame}

% %------------------------------------------------

% \begin{frame}{Blocks of Highlighted Text}
%     In this slide, some important text will be \alert{highlighted} because it's important. Please, don't abuse it.

%     \begin{block}{Block}
%         Sample text
%     \end{block}

%     \begin{alertblock}{Alertblock}
%         Sample text in red box
%     \end{alertblock}

%     \begin{examples}
%         Sample text in green box. The title of the block is ``Examples".
%     \end{examples}
% \end{frame}

% %------------------------------------------------

% \begin{frame}{Multiple Columns}
%     \begin{columns}[c] % The "c" option specifies centered vertical alignment while the "t" option is used for top vertical alignment

%         \column{.45\textwidth} % Left column and width
%         \textbf{Heading}
%         \begin{enumerate}
%             \item Statement
%             \item Explanation
%             \item Example
%         \end{enumerate}

%         \column{.45\textwidth} % Right column and width
%         Lorem ipsum dolor sit amet, consectetur adipiscing elit. Integer lectus nisl, ultricies in feugiat rutrum, porttitor sit amet augue. Aliquam ut tortor mauris. Sed volutpat ante purus, quis accumsan dolor.

%     \end{columns}
% \end{frame}

% %------------------------------------------------
% \section{Second Section}
% %------------------------------------------------

% \begin{frame}{Table}
%     \begin{table}
%         \begin{tabular}{l l l}
%             \toprule
%             \textbf{Treatments} & \textbf{Response 1} & \textbf{Response 2} \\
%             \midrule
%             Treatment 1         & 0.0003262           & 0.562               \\
%             Treatment 2         & 0.0015681           & 0.910               \\
%             Treatment 3         & 0.0009271           & 0.296               \\
%             \bottomrule
%         \end{tabular}
%         \caption{Table caption}
%     \end{table}
% \end{frame}

% %------------------------------------------------

% \begin{frame}{Theorem}
%     \begin{theorem}[Mass--energy equivalence]
%         $E = mc^2$
%     \end{theorem}
% \end{frame}

% %------------------------------------------------

% \begin{frame}{Figure}
%     Uncomment the code on this slide to include your own image from the same directory as the template .TeX file.
%     %\begin{figure}
%     %\includegraphics[width=0.8\linewidth]{test}
%     %\end{figure}
% \end{frame}

% %------------------------------------------------

% \begin{frame}[fragile] % Need to use the fragile option when verbatim is used in the slide
%     \frametitle{Citation}
%     An example of the \verb|\cite| command to cite within the presentation:\\~

%     This statement requires citation \cite{p1}.
% \end{frame}

% %------------------------------------------------

% \begin{frame}{References}
%     \footnotesize
%     \bibliography{reference.bib}
%     \bibliographystyle{apalike}
% \end{frame}

% %------------------------------------------------

% \begin{frame}
%     \Huge{\centerline{\textbf{The End}}}
% \end{frame}

% %----------------------------------------------------------------------------------------